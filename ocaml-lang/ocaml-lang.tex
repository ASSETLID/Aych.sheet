\documentclass[10pt,landscape]{article}
\usepackage{../common/cheat-style}

\begin{document}

\setlength{\premulticols}{1pt}
\setlength{\postmulticols}{1pt}
\setlength{\multicolsep}{1pt}
\setlength{\columnsep}{2em}

\makeheader{The OCaml Language}

\begin{multicols}{3}

\subsection{Syntax}

Implementations are in \Verb!.ml! files, interfaces are in \Verb!.mli!
files.\\
 Comments can be nested, between delimiters \Verb!(*!...\Verb!*)!\\
Integers: \Verb!123!, \Verb!1_000!, \Verb!0x4533!, \Verb!0o773!, \Verb!0b1010101!\\
Chars: \Verb!'a'!, \Verb!'\255'!, \Verb!'\xFF'!, \Verb!'\n'!
\hfill
Floats: \Verb!0.1!, \Verb!-1.234e-34!\\
\subsection{Data Types}

\begin{tabular}{ll}
\Verb!unit! & void, takes only one value: \Verb!()!\\
\Verb!int! & integer of either 31 or 63 bits, like \Verb!42! \\
\Verb!int32! & 32 bits Integer, like \Verb!42l! \\
\Verb!int64! & 64 bits Integer, like \Verb!42L! \\
\Verb!float! & double precision float, like \Verb!1.0! \\
\Verb!bool! & boolean, takes two values: \Verb!true! or \Verb!false!\\
\Verb!char! & simple ASCII characters, like \Verb!'A'!\\
\Verb!string! & strings, like \Verb!"Hello"! or \Verb!{foo|Hello|foo}!\\
\Verb!bytes! & mutable string of chars \\
\Verb!'a list! & lists, like \Verb!head :: tail! or \Verb![1;2;3]!   \\
\Verb!'a array! & arrays, like \Verb![|1;2;3|]! \\
$t_1$ \Verb!*! ... \Verb!*! $t_n$& tuples, like \Verb!(1, "foo", 'b')!
\end{tabular}

\subsection{Constructed Types}

\begin{tabular}{ll}
\Verb!type record =! & new record type \\
~~~ \Verb!@{!~~~~~~~~~~~~ \emph{field1} \Verb!: bool;! & immutable field \\
~~~ \Verb!  mutable! \emph{field2} \Verb!: int; @}! & mutable field \\
\end{tabular}
\begin{tabular}{ll}
\Verb!type enum =! & new variant type \\
~~~ \Verb!| Constant! & Constant constructor \\
~~~ \Verb!| Param of string! & Constructor with arg \\
~~~ \Verb!| Pair of string * int! & Constructor with args \\
~~~ \Verb!| Gadt : int -> enum! & GADT constructor \\
~~~ \Verb!| Inlined of @{ x : int @}! & Inline record \\
\end{tabular}

\subsection{Constructed Values}

\begin{Verbatim}
let r = @{ field1 = true; field2 = 3; @}
let r' = @{ r with field1 = false @}
r.field2 <- r.field2 + 1;
let c = Constant
let c = Param "foo"
let c = Pair ("bar",3)
let c = Gadt 0
let c = Inlined @{ x = 3 @}
\end{Verbatim}

\subsection{References, Strings and Arrays}

\begin{tabular}{ll}
\Verb!let x = ref 3! & integer reference (mutable) \\
\Verb!x := 4! & reference assignation \\
\verb&print_int !x;& & reference access \\
\Verb!s.[0]! & string char access \\
\Verb!t.(0)! & array element access \\
\Verb!t.(0) <- x! & array element modification \\
\end{tabular}

\subsection{Imports --- Namespaces}

\begin{tabular}{ll}
\Verb!open Unix! & global open \\
\Verb!let open Unix in! \emph{expr} & local open \\
\Verb!Unix.(!\emph{expr}\Verb!)! & local open \\
\end{tabular}

\vbox{
\subsection{Functions}

\begin{tabular}{ll}
\Verb!let f x =! \emph{expr} & function with one arg \\
\Verb!let rec f x =! \emph{expr} & recursive function \\
\hfill apply:& \Verb!f x! \\
\Verb!let f x y =! \emph{expr} & with two args \\
\hfill apply:& \Verb!f x y! \\
\Verb!let f (x,y) =! \emph{expr} & with a pair as arg\\
\hfill apply: & \Verb!f (x,y)! \\
\Verb!List.iter (fun x ->! \emph{expr}\Verb!) l! & anonymous function\\
\Verb!let f= function None ->! \emph{act}& function definition\\
\Verb!            | Some x ->! \emph{act}& \qquad [by cases]\\
\hfill apply: & \Verb!f (Some x)! \\
\Verb!let f ~str ~len =! \emph{expr} & with labeled args \\
\hfill apply: & \Verb!f ~str:s ~len:10! \\
\hfill apply (for \Verb!~str:str!): & \Verb!f ~str ~len! \\
\Verb!let f ?len ~str =! \emph{expr} & with optional arg (\Verb!option!) \\
\Verb!let f ?(len=0) ~str =! \emph{expr} & optional arg default \\
\hfill apply (with omitted arg): & \Verb!f ~str:s ! \\
\hfill apply (with commuting): & \Verb!f ~str:s ~len:12! \\
\hfill apply (\Verb!len: int option!): & \Verb!f ?len ~str:s! \\
\hfill apply (explicitly omitted): & \Verb!f ?len:None ~str:s! \\
\Verb!let f (x : int) =! \emph{expr} & arg has constrainted type \\
\Verb!let f : 'a 'b. 'a*'b -> 'a!& function with constrainted\\
\Verb!      = fun (x,y) -> x! & \hfill polymorphic type\\
\end{tabular}


\subsection{Modules}

\begin{tabular}{ll}
\Verb!module M = struct! .. \Verb!end! & module definition\\
\Verb!module M: sig! .. \Verb!end= struct! .. \Verb!end! & module and signature\\
\Verb!module M = Unix! & module renaming \\
\Verb!include M! & include items from \\
\Verb!module type Sg = sig! .. \Verb!end! & signature definition\\
\Verb!module type Sg = module type of M! & signature of module\\
\Verb!let module M = struct! .. \Verb!end in! ..  & local module \\
\Verb!let m = (module M : Sg)! & to $1^{st}$-class module\\
\Verb!module M = (val m : Sg)! & from $1^{st}$-class module\\
\Verb!module Make(S: Sg) = struct! .. \Verb!end! & functor \\
\Verb!module M = Make(M')! & functor application \\
& \\
\end{tabular}

Module type items: \Verb!val!, \Verb!external!, \Verb!type!, \Verb!exception!, \Verb!module!, \Verb!open!, \Verb!include!, \Verb!class!

\subsection{Pattern-matching}

\begin{tabular}{ll}
\Verb!match! \emph{expr} \Verb!with! \\
\Verb!  |! \emph{pattern} \Verb!->! \emph{action}\\
\Verb!  |! \emph{pattern} \Verb!when! \emph{guard} \Verb!->! \emph{action}
& conditional case \\
\Verb!  | _ ->! \emph{action} & default case\\
\end{tabular}
Patterns:\\
\begin{tabular}{ll}
\Verb!| Pair (x,y) ->! & variant pattern \\
\Verb!| @{ field = 3; _ @} ->! & record pattern \\
\Verb!| head :: tail ->! & list pattern \\
\Verb!| [1;2;x] ->! & list pattern \\
\Verb!| (Some x) as y ->! & with extra binding \\
\Verb!| (1,x) | (x,0) ->! & or-pattern \\
\Verb!| exception! \emph{exn} \Verb!->! & try\&match \\
\end{tabular}
}

\vbox{
\subsection{Conditionals}

Do NOT use on closures

\vspace{1pt}

\begin{tabular}{c|c|l}
Structural  & Physical & \\
\hline
\Verb!=!  & \Verb!==! & Polymorphic Equality \\
\Verb!<>! & \verb&!=& & Polymorphic Inequality \\
\end{tabular}

Polymorphic Generic Comparison Function: \Verb!compare!
\begin{tabular}{l|c|c|c}
                          & x $<$ y & x $=$ y & x $>$ y \\
\hline
\Verb!compare x y! &  negative    &   0   &   positive \\
\end{tabular}

Other Polymorphic Comparisons: \Verb!>!, \Verb!>=!, \Verb!<!, \Verb!<=!
}

\subsection{Loops}

\begin{Verbatim}
while cond do ... done;
for var = min_value to max_value do ... done;
for var = max_value downto min_value do ... done;
\end{Verbatim}

\subsection{Exceptions}

\begin{tabular}{p{3.5cm}p{3.5cm}}
\Verb!exception MyExn! & new exception \\
\Verb!exception MyExn of t * t'! & same with arguments\\
\Verb!exception MyFail = Failure! & rename exception with args\\
\Verb!raise MyExn! & raise an exception\\
\Verb!raise (MyExn (!\emph{args}\Verb!))! & raise with args\\
\Verb!try! \emph{expr}  & catch \Verb!MyExn!\\
\Verb!  with MyExn -> ...! & \hspace{3pt} if raised in \emph{expr} \\
\end{tabular}

\subsection{Objects and Classes}

\begin{tabular}{ll}
\Verb!class virtual foo x = !& virtual class with arg \\
\Verb! let y = x+2 in! & init before object creation\\
\Verb! object (self: 'a)! & object with self reference\\
\Verb!  val mutable variable = x! & mutable instance variable \\
\Verb!  method get = variable! & accessor \\
\Verb!  method set z =!\\
\Verb!     variable <- z+y! & mutator\\
\Verb!  method virtual copy : 'a! & virtual method\\
\Verb!  initializer! & init after object creation\\
\Verb!   self#set (self#get+1)!& \\
\Verb! end! &  \\
\Verb!class bar = !&  non-virtual class\\
\Verb! let var = 42 in! & class variable\\
\Verb! fun z -> object! & constructor argument \\
\verb& inherit foo z as super& & inheritance and ancestor reference\\
\verb& method! set y =& & method explicitly overridden\\
\Verb!    super#set (y+4)! & access to ancestor \\
\Verb! method copy = @{< x = 5 >@}! & copy with change \\
\Verb!end! & \\
\Verb!let obj = new bar 3! & new object \\
\Verb!obj#set 4; obj#get!  & method invocation \\
\Verb!let obj = object! .. \Verb!end! & immediate object \
\end{tabular}

\subsection{Polymorphic variants}


\begin{tabular}{ll}
\Verb!type t = [ `A | `B of int ]! & closed variant \\
\Verb!type u = [ `A | `C of float ]! & \\
\Verb!type v = [ t | u | ]! & union of variants \\
\Verb!let f : [< t ] -> int = function! & argument must be\\
\Verb!  | `A -> 0 | `B n -> n! & \hfill a subtype of \Verb!t!\\
\Verb!let f : [> t ] -> int = function! & \Verb!t! is a subtype \\
\Verb!  | `A -> 0 | `B n -> n | _ -> 1! & \hfill of the argument \\
\end{tabular}

\end{multicols}


\end{document}

%%% Local Variables:
%%% mode: latex
%%% TeX-master: t
%%% End:
