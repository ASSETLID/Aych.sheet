\documentclass[10pt,landscape]{article}
\usepackage{../common/cheat-style}

\begin{document}

\setlength{\premulticols}{1pt}
\setlength{\postmulticols}{1pt}
\setlength{\multicolsep}{1pt}
\setlength{\columnsep}{14pt}

\makeheader{OCaml Standard Library}

\begin{multicols}{3}

\section{Standard Modules}

\begin{itemize}
\item \verb!Stdlib!: All basic functions
\item Basic data types:
 \verb!Array!, \verb!Bool!, \verb!Bytes!, \verb!Char!, \verb!Float!, \verb!Fun!,
 \verb!Int!, \verb!Int32!, \verb!Int64!, \verb!List!, \verb!Nativeint!,
 \verb!Option!, \verb!Result!, \verb!String!, \verb!Unit!
\item Advanced data types: \verb!Bigarray!, \verb!Buffer!, \verb!Complex!,
  \verb!Digest!, \verb!Hashtbl!, \verb!Lazy!, \verb!Map!, \verb!Queue!,
  \verb!Seq!, \verb!Set!, \verb!Stack!, \verb!Stream!, \verb!Uchar!

\item System: \verb!Arg!, \verb!Filename!, \verb!Format!, \verb!Genlex!,
  \verb!Lexing!, \verb!Marshal!, \verb!Parsing!, \verb!Printexc!, \verb!Printf!,
  \verb!Random!, \verb!Scanf!, \verb!Sys!
\item Tweaking: \verb!Callback!, \verb!Ephemeron!, \verb!Gc!, \verb!Oo!, \verb!Weak!
\end{itemize}

\section{Popular Functions per Module}

\subsection{module List}
\begin{Verbacorner}
let l = {@vb{}List.init} 10 (fun i -> i)
let len = {@vb{}List.length} l
let acc' = {@vb{}List.fold@_left} (fun acc ele -> ...) acc l
let acc' = {@vb{}List.fold@_right} (fun ele acc -> ...) l acc
{@vb{}List.iter} (fun ele -> ... ) l;
{@vb{}List.iteri} (fun index ele -> ... ) l;
let l' = {@vb{}List.map}(fun ele -> ... ) l
let l' = {@vb{}List.mapi}(fun index ele -> ... ) l
let l' = {@vb{}List.filter@_map}(fun ele -> ... ) l
let l' = {@vb{}List.rev} l1
if {@vb{}List.mem} ele l then ...
if {@vb{}List.for@_all} (fun ele -> ele >= 0) l then ...
if {@vb{}List.exists} (fun ele -> ele < 0) l then ...
let x = {@vb{}List.find} (fun x -> x < 0) ints
let x_o = {@vb{}List.find@_opt} (fun x -> x < 0) ints
let negs = {@vb{}List.find@_all} (fun x -> x < 0) ints
let (negs,pos) = {@vb{}List.partition} (fun x -> x < 0) ints
let ele = {@vb{}List.nth} list 2
let head = {@vb{}List.hd} list
let tail = {@vb{}List.tl} list
let value = {@vb{}List.assoc} key assocs
if {@vb{}List.mem@_assoc} key assocs then ...
let assocs = {@vb{}List.combine} keys values
let (keys, values) = {@vb{}List.split} assocs
let l' = {@vb{}List.sort} String.compare l
let l = {@vb{}List.append} l1 l2@ @emph{or} l1 @symbol{64} l2
let list = {@vb{}List.concat} list_of_lists
\end{Verbacorner}

\begin{libcomments}
\item Functions using physical equality: \verb!memq!, \verb!assq!,
  \verb!mem_assq!
\item Non-tail recursive functions: \verb!append!, \verb!concat!, \verb!@!,
  \verb!map!, \verb!mapi!, \verb!fold_right!, \verb!map2!, \verb!fold_right2!,
  \verb!remove_assoc!, \verb!remove_assq!, \verb!split!, \verb!combine!,
  \verb!merge!
\item Raising {\vb{}Not\_found}: \verb!find!, \verb!assoc!, \verb!assq!.
\item  Raising {\vb{}Failure}: \verb!hd!, \verb!tl!, \verb!nth!.
\item Raising {\vb{}Invalid\_argument}: \verb!nth!, \verb!nth_opt!, \verb!init!,
  \verb!iter2!, \verb!map2!, \verb!rev_map2!, \verb!fold_left2!,
  \verb!fold_right2!, \verb!for_all2!, \verb!exists2!, \verb!combine!
\end{libcomments}

\columnbreak

\subsection{module Hashtbl}

\begin{Verbacorner}
let t = {@vb{}Hashtbl.create} ~random:true 117
{@vb{}Hashtbl.add} t key value;
{@vb{}Hashtbl.replace} t key value;
let value = {@vb{}Hashtbl.find} t key (* Not_found *)
let value_o = {@vb{}Hashtbl.find@_opt} t key
{@vb{}Hashtbl.iter} (fun key value -> ... ) t;
let cond = {@vb{}Hashtbl.mem} t key
{@vb{}Hashtbl.remove} t key;
{@vb{}Hashtbl.clear} t;
\end{Verbacorner}


\subsection{module String}
\begin{Verbacorner}
let s = {@vb{}String.make} len char
let len = {@vb{}String.length} s
let char = s.[pos]
let concat = prefix ^ suffix
let s' = {@vb{}String.sub} s pos len'
let s = {@vb{}String.concat} "," list_of_strings
let p' = {@vb{}String.index@_from} s p char_to_find
let p' = {@vb{}String.rindex@_from} s p char_to_find
{@vb{}String.blit} src src_pos dst_bytes dst_pos len;
let s' = {@vb{}String.uppercase@_ascii} s
let s' = {@vb{}String.lowercase@_ascii} s
let s' = {@vb{}String.escaped} s
{@vb{}String.iter} (fun c -> ...) s;
if {@vb{}String.contains} s char then ...
let l = {@vb{}String.split@_on@_char} ',' s
assert ("abc" = {@vb{}String.trim} "  abc   ");
\end{Verbacorner}

\begin{libcomments}
\item Deprecated: \verb!set!, \verb!create!, \verb!copy!, \verb!fill!,
  \verb!uppercase!, \verb!lowercase!, \verb!capitalize!, \verb!uncapitalize!
\item Raising {\vb{}Invalid\_argument}: \verb!get!, \verb!set!, \verb!create!,
  \verb!make!, \verb!init!, \verb!sub!, \verb!fill!, \verb!concat!,
  \verb!escaped!, \verb!index_from!, \verb!index_from_opt!, \verb!rindex_from!,
  \verb!rindex_from_opt!, \verb!contains_from!, \verb!rcontains_from!
\item Raising {\vb{}Not\_found}: \verb!index!, \verb!rindex!, \verb!index_from!,
  \verb!rindex_from!
\end{libcomments}

\subsection{module Bytes}
\begin{Verbacorner}
let b = {@vb{}Bytes.create} length
let b' = {@vb{}Bytes.make} length char
let b' = {@vb{}Bytes.init} length (fun i -> ...)
let b' = {@vb{}Bytes.copy} b
let b' = {@vb{}Bytes.extend} b left right
{@vb{}Bytes.blit} src srcoff dst dstoff len;
let b = {@vb{}Bytes.concat} sep blist
{@vb{}Bytes.iteri} (fun i c -> ...) b;
let s = {@vb{}Bytes.unsafe@_to@_string} b
let b = {@vb{}Bytes.unsafe@_of@_string} s
let i = {@vb{}Bytes.get@_uint8} b index
{@vb{}Bytes.set@_int32@_le} b pos 0l;
(* Bytes.[sg]et_u?int(8|(16|32|64)_[lbn]e) *)
\end{Verbacorner}

\subsection{module Array}
\begin{Verbacorner}
let t = {@vb{}Array.make} len v
let t = {@vb{}Array.init} len (fun pos -> v_at_pos)
let v = t.(pos)
t.(pos) <- v;
let len = {@vb{}Array.length} t
let t' = {@vb{}Array.sub} t pos len
let t = {@vb{}Array.of@_list} list
let list = {@vb{}Array.to@_list} t
{@vb{}Array.iter} (fun v -> ... ) t;
{@vb{}Array.iteri} (fun pos v -> ... ) t;
let t' = {@vb{}Array.map} (fun v -> ... ) t
let t' = {@vb{}Array.mapi} (fun pos v -> ... ) t
let concat = {@vb{}Array.append} prefix suffix
{@vb{}Array.sort} compare t;
\end{Verbacorner}

\subsection{modules Int, Int32, Int64, Nativeint}
\begin{Verbacorner}
module I = Int (* / Int32 / Int64 / Nativeint *)
let x = {@vb{}I.add} {@vb{}I.zero} {@vb{}I.one}
let y = {@vb{}I.mul} x ({@vb{}I.succ} x)
let d,r = {@vb{}I.div} y x, {@vb{}I.rem} y x
let x' = {@vb{}I.abs} ({@vb{}I.neg} {@vb{}I.minus@_one})
let z = {@vb{}I.shift@_left} ({@vb{}I.logor} x y) 2
let z' = {@vb{}I.shift@_right} z 2
let z' = {@vb{}I.shift@_right@_logical} z 2
(* unsigned operations not in Int *)
let c : int = {@vb{}I.unsigned@_compare} {@vb{}I.max@_int} {@vb{}I.min@_int}
let du, ru = {@vb{}I.unsigned@_div} y x, {@vb{}I.unsigned@_rem} y x
\end{Verbacorner}

\subsection{module Map}

\begin{Verbacorner}
module Dict = {@vb{}Map.Make}(String)
module Dict = {@vb{}Map.Make}(struct
    type t = String.t
    let compare = String.compare
  end)
let empty = {@vb{}Dict.empty}
let dict = {@vb{}Dict.add} "x" value_x empty
if {@vb{}Dict.mem} "x" dict then ...
let value_x = {@vb{}Dict.find} "x" dict (* Not_found *)
let value_x_o = {@vb{}Dict.find@_opt} "x" dict
let new_dict = {@vb{}Dict.remove} "x" dict
let dict' = {@vb{}Dict.update} "x"
          (function None -> ... | Some v -> ...) dict
{@vb{}Dict.iter} (fun key value -> ..) dict;
let new_dict = {@vb{}Dict.map} (fun value_x -> ..) dict
let nee_dict = {@vb{}Dict.mapi} (fun key value -> ..) dict
let acc = {@vb{}Dict.fold} (fun key value acc -> ..) dict acc
if {@vb{}Dict.equal} String.equal dict other_dict then ...
\end{Verbacorner}

\pagebreak

\subsection{module Set}

\begin{Verbacorner}
module S = {@vb{}Set.Make}(String)
module S = {@vb{}Set.Make}(struct
    type t = String.t
    let compare = String.compare end)
let empty = {@vb{}S.empty}
let set = {@vb{}S.add} "x" empty
if {@vb{}S.mem} "x" set then ...
let new_set = {@vb{}S.remove} "x" set
{@vb{}S.iter} (fun key -> ..) set;
let union = {@vb{}S.union} set1 set2
let intersection = {@vb{}S.inter} set1 set2
let difference = {@vb{}S.diff} set1 set2
let min = {@vb{}S.min@_elt} set
let max = {@vb{}S.max@_elt} set
\end{Verbacorner}

\subsection{module Char}

\begin{Verbacorner}
let ascii_65 = {@vb{}Char.code} 'A'
let char_A = {@vb{}Char.chr} 65
let c' = {@vb{}Char.lowercase@_ascii} c
let c' = {@vb{}Char.uppercase@_ascii} c
let s = {@vb{}Char.escaped} c
\end{Verbacorner}

\subsection{module Buffer}

\begin{Verbacorner}
let b = {@vb{}Buffer.create} 10_000
{@vb{}Printf.bprintf} b "Hello %s\n" name;
{@vb{}Buffer.add@_string} b s;
{@vb{}Buffer.add@_char} b '\n';
{@vb{}Buffer.add@_utf@_8@_uchar} b uc;
let s = {@vb{}Buffer.contents} b
\end{Verbacorner}




\subsection{module Digest}
\begin{Verbacorner}
let md5sum = {@vb{}Digest.string} str
let md5sum = {@vb{}Digest.substring} str pos len
let md5sum = {@vb{}Digest.file} filename
let md5sum = {@vb{}Digest.channel} ic len
let hexa = {@vb{}Digest.to@_hex} md5sum
\end{Verbacorner}



\subsection{module Filename}

\begin{Verbacorner}
if {@vb{}Filename.check@_suffix} name ".c" then ...
let file = {@vb{}Filename.chop@_suffix} name ".c"
let file_o = {@vb{}Filename.chop@_suffix@_opt} ~suffix:".c" name
let file = {@vb{}Filename.basename} name
let dir = {@vb{}Filename.dirname} name
let name = {@vb{}Filename.concat} dir file
if {@vb{}Filename.is@_relative} file then ...
let file = {@vb{}Filename.temp@_file} prefix suffix
let file = {@vb{}Filename.temp@_file} ~temp_dir:"." pref suf
\end{Verbacorner}

\columnbreak

\subsection{module Seq}

\begin{Verbacorner}
let s = {@vb{}List.to_seq} l (* works with most containers *)
let s = {@vb{}Array.to_seqi} a
(* lazy functions *)
let s = {@vb{}Seq.map} (fun e -> e) s
let s = {@vb{}Seq.filter} (fun e -> true) s
let s = {@vb{}Seq.filter_map} (fun e -> Some e) s
let s = {@vb{}Seq.flat_map} (fun e -> {@vb{}Seq.return} e) s
(* immediate functions *)
let acc = {@vb{}Seq.fold_left} (fun acc e -> ...) acc s
{@vb{}Seq.iter} (fun () -> ...) s;
(* getting one value, (recalculating) *)
let open Seq in
  match s () with
  | {@vb{}Nil} -> None
  | {@vb{}Cons} (e,s_tl) -> Some e
\end{Verbacorner}

\subsection{module Random}

\begin{Verbacorner}
{@vb{}Random.self@_init} ();
{@vb{}Random.init} int_seed;
let int_0_99 = {@vb{}Random.int} 100
let coin = {@vb{}Random.bool} ()
let float = {@vb{}Random.float} 1_000.
\end{Verbacorner}

\subsection{module Printexc}

\begin{Verbacorner}
(* $OCAMLRUNPARAM=b *)
let s = {@vb{}Printexc.to@_string} exn
let s = {@vb{}Printexc.get@_backtrace} ()
{@vb{}Printexc.register@_printer} (function
     MyExn s -> Some (Printf.sprintf ...)
   | _ -> None);
{@vb{}Printexc.set@_uncaught@_exception@_handler}
  (fun e raw_b -> () );
try ... with e ->
   let b = {@vb{}Printexc.get@_raw@_backtrace} () in
   ...
   {@vb{}Printexc.raise@_with@_backtrace} e b
\end{Verbacorner}

% $ go home emacs you're drunk

\subsection{module Ephemeron}

\begin{Verbacorner}
module E1 = {@vb{}Ephemeron.K1}
let e = {@vb{}E1.create} ()
{@vb{}E1.set@_key} e k;
{@vb{}E1.set@_data} e d;
{@vb{}E1.blit@_key} e_from e_to;
if {@vb{}E1.check@_data} e then ...
module EHASH = {@vb{}E1.Make} ( Hashable )
\end{Verbacorner}

\subsection{module Lazy}

\begin{Verbacorner}
let lazy_v = lazy (f x)
let f_x = {@vb{}Lazy.force} lazy_v
let f_x = match lazy_v with lazy f_x -> f_x
\end{Verbacorner}


\subsection{module Arg}

\begin{Verbacorner}
let arg_list = [
  "-do", {@vb{}Arg.Unit} (fun () -> ..), ": call with unit";
 "-n", {@vb{}Arg.Int} (fun int -> ..), "<n> : with int";
 "-s", {@vb{}Arg.String} (fun s -> ..), "<s> : with string";
 "-yes", {@vb{}Arg.Set} flag_ref, ": set ref";
 "-no", {@vb{}Arg.Clear} flag_ref, ": clear ref";
]
let arg_usage = "prog [args] anons: run prog with args"
{@vb{}Arg.parse} arg_list (fun anon -> ... ) arg_usage;
{@vb{}Arg.parse@_dynamic}
    (ref arg_list) (fun anon -> ...) arg_usage;
{@vb{}Arg.usage} arg_list arg_usage;
let arg_list = {@vb{}Arg.align} arg_list
\end{Verbacorner}

\subsection{module Printf}

\begin{Verbacorner}
{@vb{}Printf.printf} "flush\n%!";
let s = {@vb{}Printf.sprintf} "%s=%d or %x\n" string int hexa
{@vb{}Printf.fprintf} oc "Error: %dL=%dl\n" int64 int32;
{@vb{}Printf.bprintf} buf "%.3f if %b" float boolean;
{@vb{}Printf.printf} "%a" (fun oc x -> ...) x;
\end{Verbacorner}

\subsection{module Format}

\begin{topleft}
Formatters:\\
\begin{tabular}{lp{4cm}}
  \verb!@[<h> ...@]! & Horizontal box:\\
  & everything on one line\\
  \verb!@[<v> ...@]! & Vertical box:\\
  & next line at every break hint\\
\verb!@[<hv> ...@]! & Switch from horizontal box to vertical box if needed \\
\verb!@[<b> ...@]! & Indented box\\
\verb!@[<hov> ...@]! & Fill line after line \\
\verb!@[<... 5> ... @]! & Next line in box indented by 5 \\
\verb!@!\textvisiblespace & Breakable space \\
\verb!@,! & Break hint \\
\verb!@;! & Full break \\
\verb!@?! & Flush \\
%\verb!@\n! & Newline \\
\verb!@<4>%i! & Print int on 4 chars\\
\verb!@.! & Close everything and flush \\
\end{tabular}
\end{topleft}

\subsection{module Bigarray}

\begin{Verbacorner}
module B1 = {@vb{}Bigarray.Array1}
let a = {@vb{}B1.create} {@vb{}Bigarray.char} {@vb{}Bigarray.c@_layout} length
let e = {@vb{}B1.get} a i
{@vb{}B1.set} a i e;
{@vb{}B1.blit}
  ({@vb{}B1.sub@_left} src src_ofs len)
  ({@vb{}B1.sub@_left} dst dst_ofs len);
\end{Verbacorner}

\end{multicols}
\end{document}




% \subsection{module Queue}

% \begin{Verbacorner}
% \end{Verbacorner}



% \subsection{module Stack}
% \begin{Verbacorner}2	{@vb{}Stack.push}
% 2	{@vb{}Stack.clear}
% 1	{@vb{}Stack.pop}
% 1	{@vb{}Stack.length}
% 1	{@vb{}Stack.create}
% \end{Verbacorner}




% \subsection{module Stream}

% \begin{Verbacorner}
% \end{Verbacorner}









% \subsection{module Genlex}

% \begin{Verbacorner}
% \end{Verbacorner}





% \subsection{module Lexing}

% \begin{Verbacorner}
% 20	{@vb{}Lexing.lexeme}
% 13	{@vb{}Lexing.lexeme@_char}
% 3	{@vb{}Lexing.lexeme@_start}
% 3	{@vb{}Lexing.engine}
% 3	{@vb{}Lexing.dummy@_pos}
% 2	{@vb{}Lexing.sub@_lexeme}
% 2	{@vb{}Lexing.new@_engine}
% 2	{@vb{}Lexing.lexeme@_end}
% 2	{@vb{}Lexing.from@_channel}
% 1	{@vb{}Lexing.sub@_lexeme@_opt}
% \end{Verbacorner}



% \subsection{module Parsing}

% \begin{Verbacorner}
% 819	{@vb{}Parsing.peek@_val}
% 4	{@vb{}Parsing.yyparse}
% 2	{@vb{}Parsing.symbol@_start@_pos}
% 2	{@vb{}Parsing.symbol@_end@_pos}
% 2	{@vb{}Parsing.is@_current@_lookahead}
% 1	{@vb{}Parsing.rhs@_start@_pos}
% 1	{@vb{}Parsing.rhs@_end@_pos}
% 1	{@vb{}Parsing.parse@_error}
% 1	{@vb{}Parsing.clear@_parser}
% \end{Verbacorner}





% \subsection{module Printf}

% \begin{Verbacorner}
% 32	{@vb{}Printf.fprintf}
% 17	{@vb{}Printf.sprintf}
% 2	{@vb{}Printf.eprintf}
% 1	{@vb{}Printf.printf}
% \end{Verbacorner}



% \subsection{module Scanf}

% \begin{Verbacorner}
% \end{Verbacorner}



% \subsection{module Sys}

% \begin{Verbacorner}
% 8	{@vb{}Sys.file@_exists}
% 5	{@vb{}Sys.os@_type}
% 4	{@vb{}Sys.getenv}
% 2	{@vb{}Sys.command}
% 1	{@vb{}Sys.remove}
% 1	{@vb{}Sys.ocaml@_version}
% 1	{@vb{}Sys.getcwd}
% 1	{@vb{}Sys.executable@_name}
% \end{Verbacorner}





% \subsection{module Int32}

% \begin{Verbacorner}
% \end{Verbacorner}



% \subsection{module Int64}

% \begin{Verbacorner}
% \end{Verbacorner}



% \subsection{module Nativeint}

% \begin{Verbacorner}
% \end{Verbacorner}




% \subsection{module Complex}

% \begin{Verbacorner}
% \end{Verbacorner}



% \subsection{module Pervasives}

% (in ocaml-lang.pdf ?)

% \begin{Verbacorner}
% \end{Verbacorner}




% \end{multicols}
% \end{document}

%%% Local Variables:
%%% mode: latex
%%% TeX-master: t
%%% End:
